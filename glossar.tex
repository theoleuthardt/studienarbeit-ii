% Befehle für Abkürzungen
\newacronym{KI}{KI}{Künstliche Intelligenz}
\newacronym{UI}{UI}{User Interface}
%Eine Abkürzung mit Glossareintrag
%\newacronym{RADAR}{RADAR}{Radio Detection And Ranging\protect\glsadd{glos:radar}}

% Befehle für Glossar
\newglossaryentry{glos:nebenläufigkeit}{
    name=Nebenläufigkeit,
    description={Zwei Prozesse A und B heißen nebenläufig (concurrent), wenn sie voneinander unabhängig bearbeitet werden können. Für die Realisierung der Bearbeitung von A und B gibt es also die folgenden drei Fälle. 1. Zuerst A, dann B. 2. Zuerst B, dann A. 3. A und B gleichzeitig.\cite[S. 224]{Wagenknecht_2004}}
}
\newglossaryentry{glos:verteiltes_system}{
    name=Verteiltes System,
    description={Allgemein versteht man unter einem Verteilten System ein System, bei dem eine Reihe einzelner Funktionseinheiten, die miteinander über ein Transportsystem verbunden sind, in Zusammenarbeit Anwendungen bewältigen. \cite[S.18]{Bengel_2014}}
}
\newglossaryentry{glos:nas}{
    name=NAS,
    description={Network Attached Storage (NAS) ist ein zentraler Server, der es mehreren Benutzern ermöglicht, Dateien über ein TCP/IP-Netzwerk (Transmission Control Protocol/Internet Protocol) per WLAN oder Ethernet-Kabel zu speichern und gemeinsam zu nutzen. \cite{ibmWhatNetwork}}
}
\newglossaryentry{glos:IP}{
    name=Internetprotokoll,
    description={IP ist ein ungesichertes Vermittlungsprotokoll, das die Daten paketweise als Datagramme von der Quelle zum Ziel überträgt. Dabei wird ein Best-Effort-Ansatz gewählt. Datagramme können auch verloren gehen. Man spricht also von einem ungesicherten Schicht-3-Dienst, den IP bereitstellt. \cite[S. 10]{Mandl_2024}}
}

% wird auch in der alten Vorlage nicht benutzt?
% Befehle für Symbole
%%%%%%%%%%%%%%%%%%%%%
%\newglossaryentry{symb:Pi}{
%name=$\pi$,
%description={Die Kreiszahl.},
%sort=symbolpi, type=symbolslist
%}
%\newglossaryentry{symb:Phi}{
%name=$\varphi$,
%description={Ein beliebiger Winkel.},
%sort=symbolphi, type=symbolslist
%}
%\newglossaryentry{symb:Lambda}{
%name=$\lambda$,
%description={Eine beliebige Zahl, mit der der nachfolgende Ausdruck
%multipliziert wird.},
%sort=symbollambda, type=symbolslist
%}
%%%%%%%%%%%%%%%%%%%%%%