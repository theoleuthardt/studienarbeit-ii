
%% Vorlage Bachelorarbeit

%% Versionshistorie:

%% v1.0: Erstellung durch Johannes Woske, IT2010, j.woske+latex@gmail.com
%% v2.0: Überarbeitung und Ergänzung durch Anne Traulsen, IT2015, a.traulsen+latex@gmail.com
%% v3.0: Überarbeitung und Ergänzung durch Maja Günther, IT23B, schreibt mir keine Email LG

\documentclass[12pt, a4paper, listof=totoc, bibliography=totoc, numbers=noenddot, ngerman, headsepline, oneside]{scrbook}
\usepackage{amsmath}
\usepackage[T1]{fontenc}
\usepackage{float}
\usepackage[utf8]{inputenc}
\usepackage[ngerman]{babel}
\usepackage{url}
\usepackage{graphicx} 
\usepackage{pdfpages} 
\usepackage{multirow}
\usepackage[a4paper, margin=1in]{geometry}
\usepackage[right]{eurosym} %Euro-Zeichen
\usepackage{amssymb}
\usepackage{subfig}
\usepackage{cite}       % Quellenangaben
\usepackage{setspace}   % Zeilenabstand
\usepackage[ 
   colorlinks,       
   linkcolor=black,   % Farbe interner Verweise 
   filecolor=black,   % Farbe externer Verweise 
   citecolor=black,   % Farbe von Zitaten 
   urlcolor=blue      % Farbe von Links
   ]{hyperref}        %Verlinkungen
\usepackage[figure]{hypcap}
\usepackage[ngerman]{translator}
\usepackage{blindtext} % Lorem-Ipsum-Plugin
\usepackage[acronym, nonumberlist]{glossaries} %% use after hyperref %Glossar-Paket laden
%\usepackage[
%	nonumberlist, %keine Seitenzahlen anzeigen
%	acronym,      %ein Abkürzungsverzeichnis erstellen
%	toc,          %Einträge im Inhaltsverzeichnis
%	section       %im Inhaltsverzeichnis auf section-Ebene erscheinen
%	]
%{glossaries}

\usepackage{listings,xcolor} % Codeanzeige
\usepackage{scrhack}
\usepackage[normalem]{ulem}
\useunder{\uline}{\ul}{}
\usepackage{wrapfig}

\usepackage{makecell}
\usepackage{comment}

% ich würd immer serifenlose Schrift nehmen
% weil wer kann schon Fließtexte in Serifenschrift lesen
% aber es gibt keine offiziellen Vorgaben
\renewcommand{\familydefault}{\sfdefault} % SCHRIFTART, for reference: https://www.overleaf.com/learn/latex/Font_sizes%2C_families%2C_and_styles#Font_families 

\usepackage{pifont}
\usepackage[hashEnumerators,smartEllipses]{markdown}

\usepackage{chngcntr}
\counterwithout{figure}{chapter}
\counterwithout{table}{chapter}

\definecolor{dkgreen}{rgb}{0,.6,0}
\definecolor{dkblue}{rgb}{0.247, 0.318, 0.710}
\definecolor{dkyellow}{rgb}{204, 255, 0}
% hier beginnen lststyle spezifische Farben
\definecolor{lila}{rgb}{0.847, 0.624, 0.855}
\definecolor{hintergrund}{rgb}{0.118, 0.118, 0.118}
\definecolor{blau1}{rgb}{0.573, 0.820, 0.925}
\definecolor{orange}{rgb}{0.839, 0.616, 0.522}
\definecolor{psrot}{rgb}{0.66, 0.18, 0.00}
\definecolor{psblau}{rgb}{0, 0, 0.545}
\definecolor{psbg}{rgb}{0.9176, 0.9490, 0.9804}
\definecolor{psbg2}{rgb}{0.949, 0.949, 0.949}
\definecolor{psbg3}{rgb}{1.0, 0.9843, 0.9608}
\definecolor{key}{rgb}{1.0, 0.573, 0.0}
\definecolor{sep}{rgb}{0.424, 0.188, 0.780}
\definecolor{value}{rgb}{0.231, 0.659, 0.886}
\definecolor{goKeyword}{RGB}{0,0,255}        % Blau für Keywords
\definecolor{goBuiltin}{RGB}{102,14,122}     % Lila für Built-ins
\definecolor{goString}{RGB}{48,128,226}      % Royal Blau für Strings
\definecolor{goComment}{RGB}{121,125,129}    % Grau für Kommentare
\definecolor{goNumber}{RGB}{25,23,124}       % Dunkelblau für Zahlen
\definecolor{goBackground}{RGB}{250,250,250} % Heller Hintergrund
\definecolor{goFrame}{RGB}{200,200,200}      % Rahmenfarbe

\lstset{
    numbers=left, 
    numberstyle=\tiny\color{black}, 
    numbersep=5pt,
    breaklines=true,
    frame=lr,
    escapeinside={(*@}{@*)}, %nicht anzuzeigende Ausdrücke, z.B. für Labels
    %language=[Sharp]C,
    showstringspaces=false,
    captionpos=b,
    backgroundcolor=\color{hintergrund},
    basicstyle=\ttfamily\fontsize{9}{10}\selectfont\color{white},
    keywordstyle    = \color{dkblue},
    stringstyle     = \color{orange},
    identifierstyle = \color{blau1},
    commentstyle    = \color{dkyellow},
    emph            =[1]{var},
    emphstyle       =[1]\color{dkblue},
    emph            =[2]{if,and,or,else, return},
    emphstyle       =[2]\color{lila}
    }    
% mehr Infos zu eigenen Formatierungen: https://de.overleaf.com/learn/latex/Code_listing

\lstdefinestyle{cli}{
    basicstyle=\ttfamily\fontsize{9}{10}\selectfont\color{white},
    identifierstyle = \color{white},
    numbers=none,
    language=sh,
    stringstyle=\color{dkblue},
    emph            =[1]{dotnet},
    emphstyle       =[1]\color{dkyellow},
}
\lstdefinestyle{ps}{
    numbers=left, 
    numberstyle=\tiny\color{black}, 
    numbersep=5pt,
    breaklines=true,
    frame=none,
    escapeinside={(*@}{@*)}, %nicht anzuzeigende Ausdrücke, z.B. für Labels
    language=sh,
    showstringspaces=false,
    captionpos=b,
    backgroundcolor=\color{psbg3},
    basicstyle=\ttfamily\fontsize{9}{10}\selectfont\color{black},
    keywordstyle    = \color{psblau},
    stringstyle     = \color{psrot},
    identifierstyle = \color{psrot},
    commentstyle    = \color{dkyellow},
    emph            =[1]{var},
    emphstyle       =[1]\color{dkblue},
    emph            =[2]{if,and,or,else, return},
    emphstyle       =[2]\color{lila}
}
\lstdefinelanguage{Go}{
  morekeywords={package,import,func,var,const,type,struct,interface,map,chan,if,else,for,range,switch,case,default,go,defer,return,break,continue,fallthrough,goto,select},
  morekeywords=[2]{bool,byte,rune,string,int,int8,int16,int32,int64,uint,uint8,uint16,uint32,uint64,float32,float64,complex64,complex128,error},
  morekeywords=[3]{make,new,len,cap,append,copy,delete,close,panic,recover,print,println},
  morekeywords=[4]{true,false,nil,iota},
  keywordstyle=\color{goKeyword}\bfseries,
  keywordstyle=[2]\color{goBuiltin}\bfseries,
  keywordstyle=[3]\color{goBuiltin},
  keywordstyle=[4]\color{goBuiltin}\bfseries,
  sensitive=true,
  morecomment=[l]{//},
  morecomment=[s]{/*}{*/},
  commentstyle=\color{goComment}\itshape,
  morestring=[b]",
  morestring=[b]',
  morestring=[b]`,
  stringstyle=\color{goString},
  literate=*{0}{{{\color{goNumber}0}}}{1}
           {1}{{{\color{goNumber}1}}}{1}
           {2}{{{\color{goNumber}2}}}{1}
           {3}{{{\color{goNumber}3}}}{1}
           {4}{{{\color{goNumber}4}}}{1}
           {5}{{{\color{goNumber}5}}}{1}
           {6}{{{\color{goNumber}6}}}{1}
           {7}{{{\color{goNumber}7}}}{1}
           {8}{{{\color{goNumber}8}}}{1}
           {9}{{{\color{goNumber}9}}}{1}
}
\lstdefinestyle{golang}{
  language=Go,
  basicstyle=\ttfamily\footnotesize\color{black},
  keywordstyle=\color{goKeyword}\bfseries,      % Dezentes Dunkelblau
  stringstyle=\color{goString},                 % Dunkelgrün für Strings
  commentstyle=\color{goComment}\itshape,       % Grau für Kommentare  
  identifierstyle=\color{black},                % Schwarz für normale Bezeichner
  backgroundcolor=\color{goBackground},
  frame=single,
  rulecolor=\color{goFrame},
  framesep=8pt,
  numbers=left,
  numberstyle=\tiny\color{gray},
  numbersep=12pt,
  breaklines=true,
  breakatwhitespace=true,
  showspaces=false,
  showstringspaces=false,
  showtabs=false,
  tabsize=4,
  captionpos=b,
  columns=flexible,
  keepspaces=true,
  escapeinside={(*@}{@*)},
}


% wenn Arbeit auf Englisch -> ändern
\renewcommand\lstlistingname{Codeausschnitt}
\renewcommand\lstlistlistingname{Codeverzeichnis}

% Seitenabstände definieren
\geometry{verbose,tmargin=2cm,bmargin=2cm,lmargin=2cm,rmargin=2.3cm} 

\clubpenalty = 10000 \widowpenalty = 10000 \displaywidowpenalty = 10000 

\newcommand{\footfigref}[1]{\footnote{Abb. \ref{#1} auf Seite \pageref{#1}}}

% wenn Arbeit auf Englisch -> zu chapter ändern
\addto\extrasngerman{%
    \def\sectionautorefname{Kapitel}%
    \def\subsectionautorefname{Kapitel}%
    \def\subsubsectionautorefname{Kapitel}%
    }

% Vertikaler Abstand zwischen Ende Textblock - Ende Fußzeile -> Abstand der Seitenzahl von Rand erhöhen 
\setlength{\footskip}{10mm}

\RedeclareSectionCommand[%
    beforeskip=0.5\baselineskip,
    afterskip=0.5\baselineskip
]{chapter}

\RedeclareSectionCommand[%
    beforeskip=0.5\baselineskip,
    afterskip=0.5\baselineskip
]{section}

\RedeclareSectionCommand[%
    beforeskip=0.1\baselineskip,
    afterskip=0.1\baselineskip
]{subsection}

\RedeclareSectionCommand[%
    beforeskip=0.1\baselineskip,
    afterskip=0.1\baselineskip
]{subsubsection}

\RedeclareSectionCommand[%
    beforeskip=0.01\baselineskip,
    %%afterskip=0.2\baselineskip
]{paragraph}

\setlength{\abovecaptionskip}{4pt}  % 1pc=12pt 
\setlength{\belowcaptionskip}{0pt}
%\setlength{\textfloatsep}{4pt}
\setlength{\intextsep}{1pc}

% Verkleinerung der Textgröße unter Abbildungen
\addtokomafont{caption}{\small}

% bei falscher automatischer Silbentrennung wieder einfügen
%\include{hyphenation}

\renewcommand*{\glspostdescription}{}
 
\KOMAoptions{parskip=full*}

% ändert Titelschriftart in Serifen-Normalschriftart
\addtokomafont{disposition}{\rmfamily} 

\makenoidxglossaries

\loadglsentries{glossar.tex}

\newcommand{\type}{Studienarbeit II} % oder Bachelorarbeit
\newcommand{\topic}{Entwicklung eines interaktiven Verbindungsmanagementsystems}
\newcommand{\subtopic}{Erweiterung der Studienarbeit I mit Handshake-Protokollen für dezentrale Go-Anwendungen}
\newcommand{\studentName}{Alexander Betke, Theo Leuthardt} % mehrere Autoren: {Vorname Name, Vorname Name} etc
\newcommand{\matrikelNr}{77203378972, 77205844868}   % mehrere Autoren: {7220XXXXXXX, 7220XXXXXXX}
\newcommand{\company}{Polizei Berlin, Bundesdruckerei GmbH}
\newcommand{\jahrgang}{2023}
\newcommand{\fachbereich}{Duales Studium Wirtschaft · Technik}
\newcommand{\studiengang}{Informatik}
\newcommand{\betreuerHS}{(Prof. Dr.) Arthur Zimmermann}
\newcommand{\wordCount}{6000}


\begin{document}

\author{}
\subject{\type}
\title{
\vspace{-1em}
\normalfont\endgraf\rule{\textwidth}{1pt}\par
\vspace{0.5em}
\begingroup
	\centering
	\linespread{1.5}
	\huge\topic
\endgroup
\\
\vspace{0.1em} % Falls kein Subtopic, auskommentieren
\large\subtopic % Falls kein Subtopic, auskommentieren
\normalfont\endgraf\rule{\textwidth}{1pt}
}

\date{\vspace{-2em}\large vorgelegt am 17. Februar 2026}
% oder wenn ihr das hier schöner findet:
%%%%%%%%%%%%%%%%%%%%%%%%%%%%%%%%%%%%%%%%
% \date{\normalsize vorgelegt am 18. August 2025\\ \textbullet \\ Fachbereich Duales Studium Wirtschaft · Technik \\
% Hochschule für Wirtschaft und Recht Berlin}
%%%%%%%%%%%%%%%%%%%%%%%%%%%%%%%%%%%%%%%%

\publishers{
	\begin{tabular}{l l}
	\textbf{\normalsize{}} & \normalsize{}  \tabularnewline
	\textbf{\normalsize{}} & \normalsize{}  \tabularnewline
	\textbf{\normalsize{Name:}} & \normalsize{\studentName}  \tabularnewline
 	\textbf{\normalsize{Matrikelnummer:}} & \normalsize{\matrikelNr}  \tabularnewline
	\textbf{\normalsize{Ausbildungsbetrieb:}} & \normalsize{\company}  \tabularnewline
	\textbf{\normalsize{Studienjahrgang:}} & \normalsize{\jahrgang}  \tabularnewline
	\textbf{\normalsize{Fachbereich:}} & \normalsize{\fachbereich} \tabularnewline
	\textbf{\normalsize{Studiengang:}} & \normalsize{\studiengang} \tabularnewline
	\textbf{\normalsize{Betreuer/in Hochschule:}} & \normalsize{\betreuerHS} \tabularnewline
    \textbf{\normalsize{Anzahl der Wörter:}} & \normalsize{\wordCount} \tabularnewline
    \tabularnewline
	\end{tabular}
    % zwe Autoren Autor:
    \begin{tabular}{p{15em} p{1em} p{8em}}
        \normalsize{Vom Ausbildungsleiter zur Kenntnis genommen:} \tabularnewline
        \tabularnewline
        \hspace{4cm} && \hspace{4cm}\\\cline{1-1}\cline{3-3}
        \normalsize{AL Cynthia Reuter} && \normalsize{Alexander Betke}
        \tabularnewline
        \hspace{4cm} && \hspace{4cm}\\\cline{1-1}\cline{3-3}
        \normalsize{AL Sven Krausch} && \normalsize{Theo Leuthardt} 
    \end{tabular}
    }

\titlehead{\begin{center}
    \includegraphics[height=0.04\textheight]{bilder/HWR.png}
    \hfill
    \includegraphics[height=0.04\textheight]{bilder/PolizeiBerlin.jpg} % hier DEIN Firmenlogo
    \hfill
    \includegraphics[height=0.04\textheight]{bilder/BDR.png} % hier DEIN Firmenlogo
    \end{center}
    }

\maketitle
\onehalfspacing 

% römische Seitenzahlen
\pagenumbering{Roman}

% Abstract (ausgelagert in extra File)
\chapter*{Abstract}\label{abstract}
\addcontentsline{toc}{chapter}{Abstract}

% TODO: Abstract schreiben

% Bei mehreren Autoren
%%%%%%%%%%%%%%%%%%%%%%%%
\section*{Schreibverteilung}
\begin{table}[H]
\small
\begin{tabular}{| l | l |}
\hline
Autor   & Kapitel \\ [0.5ex]
\hline
\hline
Theo Leuthardt & Abstract\\
        & Einleitung\\
        & Problemstellung\\
        & Anforderungsanalyse\\
        & Theoretische Grundlagen\\
        & Entwurf\\
        & Algorithmen\\
\hline
Alexander Betke & Diskussion\\
        & Fazit\\
\hline
\end{tabular}
\end{table}
\clearpage
%%%%%%%%%%%%%%%%%%%%%%%%
\newpage

% Inhaltsverzeichnis
\tableofcontents{}
% ändern, wenn englische Arbeit
\addcontentsline{toc}{chapter}{Inhaltsverzeichnis}
\clearpage

% ich würd die folgenden drei Verzeichnisse immer nachs KI Verzeichnis, vor den Anhang einfügen
% damit ihr nicht noch voem Text drei halbleere Seiten habt
% aber Frau Monett Diaz meinte, nach dem Inhaltsverzeichnis ist besser (:

% Abbildungsverzeichnis -> auskommentieren, wenn keine Abbildungen verwendet wurden
\listoffigures
\clearpage

% Tabellenverzeichnis -> auskommentieren, wenn keine Tabellen verwendet wurden
%\listoftables
%\clearpage

% Codeverzeichnis -> auskommentieren, wenn keine Codeausschnitte verwendet wurden
\lstlistoflistings
\clearpage

% Akronyme extra
\addcontentsline{toc}{chapter}{Akronyme}
\printnoidxglossary[type=\acronymtype]
\printnoidxglossary[title=Glossar]

\clearpage

% arabische Seitenzahlen sobalds losgeht
\pagenumbering{arabic}

% alle Kapitel schick ausgelagert
\chapter{Einleitung}\label{einleitung}

Verteilte Systeme ermöglichen die effiziente Bearbeitung komplexer Aufgaben durch eine potentiell effiziente Aufteilung auf mehrere bearbeitende Instanzen.
Bei zunehmender Komplexität solcher Aufgaben werden Protokolle benötigt, um einen sicheren Datenaustausch zwischen den einzelnen Bearbeitern zu gewährleisten.
Um die Aufteilung von Aufgaben auf Teilnehmer verteilter Systeme zu erforschen und das Leser-/Schreiber-Problem zu visualisieren wird in der vergangenen Studienarbeit I 
eine Applikation in Go entwickelt, mit der zwei Instanzen dieser Anwendung in einem lokalen Netzwerk eine Verbindung aufbauen können per TCP-Protokoll und anschließend Pakete gegenseitig aneinander versendet werden.
\\
Mit dieser Arbeit wird die bestehende Implementierung erweitert um ein Verbindungsmanagementsystem.
Dieses soll dem Nutzer die Möglichkeit geben selbst die Instanz auszuwählen, mit der sich die Anwendung verbinden soll.
Für die Erweiterung wird im Frontend eine Übersicht aktuell vorhandener und verbindungsbereiter Instanzen entwickelt zum interaktiven Aussuchen des Verbindungspartners.
Dafür wird auf Basis des TCP-Handshake-Protokolls ein eigenes Handshake-Protokoll entworfen und für saubere Verbindungstrennungen Graceful-Disconnections implementiert.
Da dies eine Erweiterung des vorhandenen Prgrammcodes ist, wird die Programmiersprache GoLang für das Frontend und Backend beibehalten.
\\
Im fogenden Kapitel werden mit der \ref{problemstellung} die noch vorhandenen Probleme aus Studienarbeit 1, die Problemrelevanz und noch fehlende Funktionalitäten beschrieben.
Die daran anschließende Anforderungsanalyse konkretisiert die funktionalen und nicht funktionalen Anforderungen, die sich daraus ergeben.
Für das Verständis der späteren Kapitel wie dem \ref{entwurf} zu architektonischen Anpassungen oder \ref{algorithmen} mit den konkreten Implementierungen wird im Kapitel \ref{grundlagen} jegliche konzeptionelle Grundlage erläutert.
Mit der \ref{diskussion} werden die Ergebnisse des Projekts evaluiert und von einem kritischen Standpunkt betrachtet.
Abschließend wird in Kapitel \ref{fazit} ein Fazit gezogen und zukünftige mögliche Änderungen bzw. Verbesserungen angeführt.

\chapter{Problemstellung}\label{problemstellung}
% Was ist das eigentliche Problem
Wie schon in Kapitel \ref{einleitung} erwähnt, wird es zum Problem in einem \gls{glos:verteiltes_system}, wenn durch zwei Prozesse versucht wird auf dieselben Daten zuzugreifen in Form von gleichzeitigem Lesen und Schreiben.
Dies ist dem Leser-Schreiber-Problem zuzuschreiben und verschiedene kritische Probleme werden dadurch verursacht: Gleichzeitiges Lesen und Schreiben derselben Daten führt zu Datenkorruption, inkonsistente Zustände werden durch unkoordinierte Zugriffe erzeugt und Performanzprobleme können daraus resultieren.
Beispielsweise wird jene Problematik im Falle eines \gls{glos:nas} in einem verteilten System aus zwei Computern in einem lokalen Netzwerk und dem \gls{glos:nas}-Server möglicherweise auftreten.
Falls durch beide Computer auf die gleiche Textdatei in Form von gleichzeitigem Lesen oder Schreiben zugegriffen wird, kann das Leser-Schreiber-Problem auftreten und Inkonsistenzen treten auf im Text der Textdatei.
So können verschiedene Wörter oder Sätze geschrieben werden durch Computer 1, die Computer 2 aktuell noch nicht auslesen kann, da das gleichzeitige Schreiben und Lesen auf die Datei nicht möglich ist. 

\section{Problemrelevanz}\label{problemrrelevanz}
% Warum ist es relevant?
Da dieses Problem wie im zuvor genannten Beispiel, durch gleichzeitig lesend und schreibende Prozesse auf demselben Computer oder in vielen anderen technischen Szenarien auftreten kann wie Cloud Computing oder bei Microservices, wird die hohe Relevanz dieses Problems für das Verständnis verteilter Systeme deutlich. 
Obwohl bereits etablierte Lösungsansätze existieren, wird eine fundierte Auseinandersetzung mit dieser Thematik als essentiell für den Einstieg in verteilte Systeme betrachtet (nähere Erläuterung des Leser-Schreiber-Problems und dessen Lösungen in Kapitel \ref{leserschreiberproblem}).

\section{Was fehlt bisher?}\label{wasfehlt?}
% Was fehlt bisher?
Wird versucht das Leser-Schreiber-Problem besser zu verstehen, vor allem durch visuell-lernende Menschen, dann werden diese nur bedingt Visualisierungen finden im Internet neben vielen mathematischen Formeln und Darstellungen.
Um die Lösungen dieses Problems greifbarer zu machen, würde es Sinn ergeben Diagramme bzw. ein Modell zu erstellen, womit gleichzeitig das Problem und auch die vorhandenen Lösungen direkt auf der Hand liegen würden.
Jedoch wird ein Mangel an Interaktivität festgestellt, die zum Lernerfolg beitragen könnte.
So würde eine Desktop-Applikation helfen, den gleichzeitigen Zugriff auf Daten zu visualisieren auf interaktive Art und Weise und die Lösung des Leser-Schreiber-Problems zu erleben.

\section{Anforderungen}\label{anforderungen}
% Konkrete Anforderungen ans Projekt/an die Anwendung
Somit wird im Rahmen dieser Arbeit eine Desktop-Applikation entwickelt, womit eine Netzwerkverbindung auf Basis des Protokolls TCP aufgebaut werden kann und TCP-Pakete erhalten und versendet werden können. 
Die Desktop-Applikation soll in ein Frontend und Backend unterteilt sein.
Das Backend soll als Server dienen, der mit anderen Instanzen der Anwendung, also mit einem Backend einer anderen Instanz der Anwendung eine Verbindung per TCP im selben lokalen Netzwerk aufbauen kann. 
Folgend sollen automatisch drei Arten von TCP-Paketen mit verschiedenen Intervallen ausgetauscht werden können: slow, dynamic und fast, die gezählt werden sollen innerhalb des Backend.
Dieser Austausch wird definiert durch das automatische und gleichzeitige Senden und Empfangen der TCP-Pakete.
Die Zählungen der Pakete sollen global in der Anwendung gespeichert und aktualisiert werden durch das Backend.
Vor dem Versenden des TCP-Pakets soll die Art des Pakets selbstständig ausgewählt werden durch das Backend.
Im Frontend sollen die drei Zähler der Paketarten aus dem globalen Speicher ausgelesen werden und die jeweilige Anzahl der erhaltenen Paketart soll in der UI der Anwendung visuell dargestellt werden in Form von Balken, die sich füllen.
Damit soll das Szenario erschaffen werden, dass jeweils Frontend und Backend der Anwendungen gleichzeitig auf den globalen Speicher zugreifen und demnach das Leser-Schreiber-Problem live auftritt bzw. eine Synchronisation als Lösung im Frontend implementiert wird.
Die Desktop-Applikation soll auf den Betriebsystemen Windows und Linux installierbar sein und auch betriebsystemübergreifend TCP-Verbindungen aufbauen können.
Als visueller Zusatz soll ein Dark- und Lightmode eingeführt werden für die \acrshort{UI} und zwischen denen durch einen Button gewechselt werden kann.
\chapter{Anforderungsanalyse}\label{anforderungsanalyse}
Mit der Erweiterung der vorhanden Implementierung aus Studienarbeit I werden Definitionen von Anforderungen gefordert.
Damit wird eine Grundlage gebildet, auf der der Entwurf und die Implementierung des Verbindungsmanagementsystems umgesetzt werden kann.
Es wird als notwendig betrachtet, dass alle Grundfunktionalitäten aus der Studienarbeit I beibehalten werden.
Dazu wird auf funktionaler Seite das automatische Erkennen von Peers im lokalen Netzwerk gezählt, sowie die Nutzung des Portbereichs 50500 bis 50600, das gegenseitige Versenden von TCP-Paketen, die Anzeige und das Monitoring des aktuellen Verbindungsstatuses, die Visualiserung des Leser-/Schreiber-Problems und die Channel-basierte Kommunikation zwischen Frontend und Backend.
Bei den nicht funktionales Anforderungen werden die Plattformunabhängigkeit der Anwendung, UI-Skalierung und der Dark-Mode als vorhanden gesehen.\\
In den folgenden Teilabschnitten werden die Anforderungen in funktionale und nicht funktionale unterteilt und darin jeweils in kategorisierte Untergruppen.
Jede definierte Anforderung wird mit einer eindeutigen Indentifikationsnummer notiert für die spätere Referenzierung in Kapitel \ref{entwurf} und \ref{algorithmen}.

\section{Funktionale Anforderungen}\label{funktionale_anforderungen}
In folgender Auflistung werden die neuen funktionalen Anforderungen beschrieben für das erwartete Verhalten der Desktop-Anwendung.

\subsection{Peer-Discovery}
\begin{description}
    \item[FA-01: Peer-Liste-Verwaltung] \hfill \label{FA-01} \\
    Zusätzlich zur automatischen Erkennung von Peers im lokalen Netzwerk soll das Backend eine Liste an erkannten Peers führen und diese regelmäßig bei jeder neuen Suche aktualisieren.

    \item[FA-02: Discovery-Handshake] \hfill \label{FA-02} \\
    Damit auch nur valide Peers erkannt werden, soll vor dem eigentlichen Verbindungs-Handshake eine Erkennungs-Handshake implementiert werden in Form von: \texttt{DISCOVER\_SYN} / \texttt{DISCOVER\_ACK}.
    Dies soll zur Prüfung dienen, ob der gefundene Peer verfügbar für eine anschließende Verbindung ist.
\end{description}

\subsection{Verbindungsverwaltung}
\begin{description}
    \item[FA-03: Manuelle Peer-Auswahl] \hfill \label{FA-03} \\
    Bei der Nutzung der Anwendung soll der Benutzer über eine grafische Benutzeroberfläche eine Übersicht über alle aktuell erkannten Peers erhalten, mit denen potentiell eine Verbindung aufgebaut werden kann.

    \item[FA-04: Peer-Auswahl-Ausblendung im verbundenen Zustand] \hfill \label{FA-04} \\
    Wird eine Verbindung hergestellt mit einem anderen Peer, so soll die Peer-Auswahl-Übersicht aus Anforderung \ref{FA-03} ausgeblendet werden.
    
    \item[FA-05: Verbindungs-Handshake-Protokoll] \hfill \label{FA-05} \\
    Für das Verbinden mit einem Peer soll ein Handshake-Protokoll implementiert werden mit \texttt{CONNECT\_REQ:<ip>:<port>} und \texttt{CONNECT\_OK} Nachrichten.

    \item[FA-06: Bidirektionale Informationsaustausch über die Verbindung] \hfill \label{FA-06} \\
    Bei einer Verbindungsanfrage über die \texttt{CONNECT\_REQ} Nachricht soll die IP-Addresse und der Port des Verbindungsinitiators mit versendet werden, damit auch der Empfänger die Information erhält, wer mit ihm eine Verbindung aufbauen möchte.

    \item[FA-07: Verbindungsexklusivität] \hfill \label{FA-07} \\
    Wird zwischen zwei Peers eine Verbindung erfolgereich aufgebaut und erhalten, sollen Verbindungsanfrage-Nachrichten anderer Peers währenddessen ignoriert werden.
\end{description}

\subsection{Verbindungstrennung}
\begin{description}
    \item[FA-08: Manueller Disconnect] \hfill \label{FA-08} \\
    Dem Benutzer der Anwendung soll es über eine Schaltfläche ermöglicht werden, eine bestehende Verbindung zu einem anderen Peer trennen zu können, ohne die Anwendung schließen zu müssen.

    \item[FA-09: Graceful-Disconnects] \hfill \label{FA-09} \\
    Wird eine bestehende Verbindung zu einem anderen Peer manuell unterbrochen durch den Benutzer, soll eine \texttt{DISCONNECT} Nachricht erst an den Peer gesendet werden für die Verbindungsauflösung.

    \item[FA-10: Zustandssynchronisation bei Disconnect] \hfill \label{FA-10} \\
    Nachdem durch den Benutzer die Verbindung zu einem anderen Peer getrennt wird, sollen beide Instanzen ihre Zustände wie der Verbindungsstatus, die Adresse des verbundenen Peers und die Queue-Werte zurücksetzen auf ihre Initialwerte.
\end{description}

\section{Nichtfunktionale Anforderungen}\label{nichtfunktionale_anforderungen}
Im Folgenden werden die nicht funktionalen Anforderungen definiert zur Hervorhebung qualitativer Eigenschaften der Anwendung.

\subsection{Performance}
\begin{description}
    \item[NFA-01: Discovery-Geschwindigkeit] \hfill \label{NFA-01} \\
    Ein vollständiger Discovery-Scan (localhost + Subnetz) soll innerhalb von maximal 5-7 Sekunden abgeschlossen sein.

    \item[NFA-02: UI-Aktualisierungsrate] \hfill \label{NFA-02} \\
    Die Benutzeroberfläche muss im Intervall von 100ms mit aktuellen Zustandsinformationen aus dem Backend aktualisiert werden.

    \item[NFA-03: Handshake-Timeout] \hfill \label{NFA-03} \\
    Discovery-Handshakes sollen einen Timeout von 2 Sekunden haben und die Verbindungs-Handshakes einen Timeout von 1 Sekunde.
\end{description}

\subsection{Benutzerfreundlichkeit}
\begin{description}
    \item[NFA-04: Intuitive Bedienung] \hfill \label{NFA-04} \\
    Die Auswahl von Verbindungspartnern und das Trennen der aktuellen Verbindung soll über jeweils einen Klick für den Benutzer erledigt werden können.

    \item[NFA-05: Identifikation des Verbindungspartners vor dem Verbindungsaufbau] \hfill \label{NFA-05} \\
    Jeder der angezeigten Schaltflächen für einen möglichen Verbindungspartner soll die IP-Adresse und den Port des Peers enthalten.
\end{description}

\subsection{Zuverlässigkeit}
\begin{description}
    \item[NFA-06: Fehlertoleranz] \hfill \label{NFA-06} \\
    Die Anwendung soll bei Netzwerkfehlern oder beim Abbruch der Verbindung automatisch in einen weiterhin konsistenten Zustand bleiben.
\end{description}

\subsection{Skalierbarkeit}
\begin{description}
    \item[NFA-08: Dynamisches Anzeige der verbindungsbereiten Peers] \hfill \label{NFA-08} \\
    Die Anzahl der Schaltflächen zur manuellen Verbindung mit anderen Peers soll nicht durch die Fenstergröße begrenzt sein.
    Das bedeutet, neue Schaltflächen sollen zur Übersicht hinzugefügt werden, jedoch soll das Fenster nicht linear mit der Anzahl der anzuzeigenden Buttons größer werden.
\end{description}
\chapter{Theoretische Grundlagen}\label{grundlagen}

% TODO: Theoretische Grundlagen schreiben
% Erklärung aller notwendigen Prinzipien für die Entwicklung der Erweiterung
% z.B. tiefere theoretische Erklärung des Findens von anderen Instanzen im lokalen Netzwerk
\chapter{Entwurf}\label{entwurf}

% TODO: Entwurf schreiben (mit Referenzen zu Anforderungen)

\section{Strukturelle Veränderungen im Backend}\label{backend-veraenderungen}

% TODO: Beschreibung der strukturellen Veränderungen im Backend

\section{Strukturelle Veränderungen im Frontend}\label{frontend-veraenderungen}

% TODO: Beschreibung der strukturellen Veränderungen im Frontend
\chapter{Algorithmen}\label{algorithmen}

% TODO: Algorithmen schreiben (mit Referenzen zu Anforderungen)
% Pro konkreter Codeänderung ein Unterabschnitt erstellen

% Beispielstruktur:
% \section{Name der konkreten Codeänderung 1}
% \section{Name der konkreten Codeänderung 2}
% usw.
\chapter{Diskussion}\label{diskussion}

% TODO: Diskussion schreiben
% Vergleich Vorher/Nachher nach der Erweiterung der Software
% Aspekte: Nutzerfreundlichkeit, Nutzbarkeit der Software
\chapter{Fazit}\label{fazit}

% TODO: Fazit schreiben

\newpage
\bibliography{literatur}
\nocite{*}  % listet ALLE Quellen in literatur.bib, auch wenn sie nicht zitiert wurden -> ergibt Sinn, auszukommentieren
\bibliographystyle{hwrbib}

% KI-Verzeichnis
\addchap{KI-Verzeichnis}\label{ch:ki}
% die alte Vorlage hat kein KI Verzeichnis
% das hier basiert auf einer Bachelorarbeit aus 2024
% aber wenn ihr sucht, findet ihr bestimmt auch Vorschriften
\begin{table}[H]
    \centering
    \begin{tabular}{|l|l|l|}
         \hline
         \textbf{\acrshort{KI}-basiertes Hilfsmittel} & \textbf{Einsatzform} & \textbf{Betroffene Teile der Arbeit} \\
         \hline
         \hline
         Generative KI (Claude) & Komprimierung von Textpassagen & \autoref{fazit}Fazit \\
         \hline
         DeepL Translator & Übersetzung von Textpassagen & \autoref{abstract}Theoretische Grundlagen \\
         \hline
         Generative KI (Claude) & Einstieg in Themen fürs Verständnis & \autoref{abstract}Theoretische Grundlagen \\
         \hline
    \end{tabular}
    \label{tab:ki}
\end{table}

% wenn euer Anhang viele Seiten hat, ergibt es Sinn, zu alphabetischen Seitenzahlen zu "wechseln"
%%%%%%%%%%%%%%%%%%%%%%
% \pagenumbering{Alph}
%%%%%%%%%%%%%%%%%%%%%%
% Anhang
%\addchap{Anhang}\label{ch:anhang}
\begin{figure}[H]

\end{figure}

% Ehrenwörtliche Erklärung
\chapter*{Ehrenwörtliche Erklärungen}
\addcontentsline{toc}{chapter}{Ehrenwörtliche Erklärung}
% Dieser Text stammt aus der Studiengangsbeschreibung von November 2024 (abgerufen April 2025)
Hiermit versichere ich, dass ich die vorliegende Arbeit in allen Teilen selbstständig angefertigt und keine anderen, als die in der Arbeit angegebenen Quellen und Hilfsmittel benutzt habe. Sämtliche wörtlichen oder sinngemäßen Übernahmen und Zitate, sowie alle Abschnitte, die mithilfe von \acrshort{KI}-basierten Tools entworfen, verfasst und/oder bearbeitet wurden, sind kenntlich gemacht und nachgewiesen.

Im Anhang meiner Arbeit habe ich sämtliche \acrshort{KI}-basierte Hilfsmittel angegeben. Diese sind mit Produktnamen und formulierten Eingaben (Prompts) in einem \acrshort{KI}-Verzeichnis ausgewiesen. 

Ich bin mir bewusst, dass die Verwendung von Texten oder anderen Inhalten und Produkten, die durch \acrshort{KI}-basierte Tools generiert wurden, keine Garantie für deren Qualität darstellt. Ich verantworte die Übernahme jeglicher von mir verwendeter maschinell generierter Passagen vollumfänglich selbst und trage die Verantwortung für eventuell durch die \acrshort{KI} generierte fehlerhafte oder verzerrte Inhalte, fehlerhafte Referenzen, Verstöße gegen das Datenschutz- und Urheberrecht oder Plagiate.
\\ \\ \\

\begin{tabular}{lp{2em}l} 
 \hspace{4cm}   && \hspace{5cm} \\\cline{1-1}\cline{3-3} 
 Ort, Datum     && \text{Alexander Betke} \\
 \\
 \\
\end{tabular}

% pro Autor ergänzen
%%%%%%%%%%%%%%%%%%%%%%%%%%%%
\begin{tabular}{lp{2em}l} 
 \hspace{4cm}   && \hspace{5cm} \\\cline{1-1}\cline{3-3} 
 Ort, Datum     && \text{Theo Leuthardt} 
\end{tabular}
%%%%%%%%%%%%%%%%%%%%%%%%%%%%

\end{document}
