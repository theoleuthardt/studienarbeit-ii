\chapter{Aufzählungen}\label{ch:aufzählungen}
Numerierung:

Kapitel (\verb|chapter|) können in
\begin{enumerate}
    \item Unterkapitel (\verb|section|) und
    \item Unter-Unterkapitel (\verb|subsection|), usw.
\end{enumerate}
gegliedert werden.
\\ 

Aufzählung:

Kapitel (\verb|chapter|) können in
\begin{itemize}
    \item Unterkapitel (\verb|section|) und
    \item Unter-Unterkapitel (\verb|subsection|), usw.
\end{itemize}
gegliedert werden.
\\

Mit \verb|vspace()| können die Abstände manuell angepasst werden:
\begin{itemize}
    \vspace{-1em}
    \item negative Werte zum verkleinern,
    \vspace{2em}
    \item positive, um den Abstand zu vergrößern.
\end{itemize}
\vspace{2em}

Man kann auf alle Elemente über ihre Label verweisen:
\begin{itemize}
    \vspace{-1em}
    \item \autoref{ch:aufzählungen}
    \vspace{-1em}
    \item \autoref{fig:single}
    \vspace{-1em}
    \item \autoref{table:tab3}
    \vspace{-1em}
    \item \autoref{lst:cli}
\end{itemize}
Ausnahmen dazu sind Kapitel, die nicht im Inhaltsverzeichnis angezeigt werden und Bilder/Codeabschnitte/Tabellen ohne Caption, die dementsprechend auch nicht im jeweiligen Verzeichnis gelistet werden.