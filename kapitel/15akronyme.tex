\chapter{Akronyme}\label{ch:akronyme}
Das Wort Akronym stammt aus dem Griechischen. Es handelt sich um ein Initialwort, dass sich in der Regel aus den Anfangsbuchstaben oder -silben von mehreren Wörtern zusammensetzt. Ein Akronym ist – vor allem im \acrshort{IT}-Bereich – oft eine Wortneuschöpfung. Aber es gibt natürlich auch ältere Akronyme. Ein Beispiel dafür ist \acrfull{RADAR}, das im Zweiten Weltkrieg entstanden sind. 

Bei der ersten Verwendung ergibt \verb|\acrfull{}| Sinn: \\
\acrfull{RADAR},

im weiteren Verlauf dann \verb|\acrshort{}|: \\
\acrshort{RADAR},

oder (für Abwechslung) \verb|\acrlong{}|: \\
\acrlong{RADAR}.

Beim Hovern über das Wort sollte es farblich hervorgehoben werden und klickt man darauf, wird man in das Akronym-Verzeichnis weitergeleitet.
\\

Zu Akronymeinträgen können \gls{glos:glossar}einträge hinzugefügt werden (siehe \verb|glossar.tex|). Auch ausgeschriebene Wörter können mit einem \gls{glos:glossar}eintrag versehen werden. Dann werden sie mit \verb|\gls{}| im Text geschrieben.