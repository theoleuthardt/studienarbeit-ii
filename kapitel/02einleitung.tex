\chapter{Einleitung}\label{einleitung}

Verteilte Systeme ermöglichen die effiziente Bearbeitung komplexer Aufgaben durch eine potentiell effiziente Aufteilung auf mehrere bearbeitende Instanzen.
Bei zunehmender Komplexität solcher Aufgaben werden Protokolle benötigt, um einen sicheren Datenaustausch zwischen den einzelnen Bearbeitern zu gewährleisten.
Um die Aufteilung von Aufgaben auf Teilnehmer verteilter Systeme zu erforschen und das Leser-/Schreiber-Problem zu visualisieren wird in der vergangenen Studienarbeit I 
eine Applikation in Go entwickelt, mit der zwei Instanzen dieser Anwendung in einem lokalen Netzwerk eine Verbindung aufbauen können per TCP-Protokoll und anschließend Pakete gegenseitig aneinander versendet werden.
\\
Mit dieser Arbeit wird die bestehende Implementierung erweitert um ein Verbindungsmanagementsystem.
Dieses soll dem Nutzer die Möglichkeit geben selbst die Instanz auszuwählen, mit der sich die Anwendung verbinden soll.
Für die Erweiterung wird im Frontend eine Übersicht aktuell vorhandener und verbindungsbereiter Instanzen entwickelt zum interaktiven Aussuchen des Verbindungspartners.
Dafür wird auf Basis des TCP-Handshake-Protokolls ein eigenes Handshake-Protokoll entworfen und für saubere Verbindungstrennungen Graceful-Disconnections implementiert.
Da dies eine Erweiterung des vorhandenen Prgrammcodes ist, wird die Programmiersprache GoLang für das Frontend und Backend beibehalten.
\\
Im fogenden Kapitel werden mit der \ref{problemstellung} die noch vorhandenen Probleme aus Studienarbeit 1, die Problemrelevanz und noch fehlende Funktionalitäten beschrieben.
Die daran anschließende Anforderungsanalyse konkretisiert die funktionalen und nicht funktionalen Anforderungen, die sich daraus ergeben.
Für das Verständis der späteren Kapitel wie dem \ref{entwurf} zu architektonischen Anpassungen oder \ref{algorithmen} mit den konkreten Implementierungen wird im Kapitel \ref{grundlagen} jegliche konzeptionelle Grundlage erläutert.
Mit der \ref{diskussion} werden die Ergebnisse des Projekts evaluiert und von einem kritischen Standpunkt betrachtet.
Abschließend wird in Kapitel \ref{fazit} ein Fazit gezogen und zukünftige mögliche Änderungen bzw. Verbesserungen angeführt.
