\chapter{Einleitung}\label{einleitung}
Verteilte Systeme bilden heutzutage das Fundament moderner Softwarearchitekturen und ermöglichen die effiziente Bearbeitung komplexer Aufgaben durch koordinierte Aufteilung auf mehrerer Instanzen.
Durch die wachsende Größe an zu berechnenden Aufgaben solcher Softwaresysteme werden Protokolle verwendet, um Daten zwischen Bearbeitern in verteilten Systemen austauschen zu können.
In der vorherigen Studienarbeit I wird ein System in Form einer Applikation entworfen und implementiert, womit innerhalb eines verteilten Systems wie einem lokalen Netzwerk zwei Instanzen 
automatisch eine Verbindung herstellen und Pakete austauschen können. 
Mithilfe dieser Applikation wird es ermöglicht das Leser-Schreiber-Problem zu visualisieren und dadurch aufkommende Wartezeiten sichbar zu machen.\\
\\
Mit dieser Arbeit wird die bestehende Implementierung der Applikation aus Studienarbeit I erweitert mit einem Verbindungsmanagementsystem zur manuellen Auswahl eines Verbindungspartners.
Dabei wird eine Trennung zwischen automatischer Peer-Discovery im Backend und manueller Verbindungsauswahl im Frontend geschaffen.
Um die das Verbindungsmanagementsystem zu realisieren, werden eigene Handshake-Protkolle und Graceful-Disconnections implementiert.
Als Programmiersprache wird weiterhin Go eingesetzt, um auf Basis des vorhandenen Codes die praktische Umsetzung durchzuführen.\\
\\
Im folgenden Kapitel wird die Problemstellung erläutert zunächst eine detaillierte Analyse der Schwachstellen der bisherigen Lösung und der daraus resultierenden Anforderungen an die Erweiterung.
Die Anforderungsanalyse strukturiert anschließend die funktionalen sowie nichtfunktionalen Kriterien, die das neue Verbindungsmanagementsystem erfüllen muss.
Die benötigten konzeptionellen Grundlagen der Peer-Discovery-Verfahren sowie der entwickelten Handshake-Mechanismen werden im Kapitel Theoretische Grundlagen vermittelt.
Im Entwurf werden die architektonischen Anpassungen sowohl im Backend als auch im Frontend vorgestellt, wobei der Fokus auf der Separation zwischen Netzwerkerkennung und Verbindungssteuerung liegt.
Die praktische Umsetzung der einzelnen Komponenten wird im Kapitel Algorithmen behandelt, das ebenfalls in Frontend und Backend unterteilt wird.
Folgend findet eine vergleichende Betrachtung der ursprünglichen und erweiterten Softwareversion statt hinsichtlich Bedienbarkeit und Funktionsumfang.
Abchließend wird das Fazit gebildet mit einer Zusammenfassung der Erkenntnisse und einem Ausblick auf potenzielle Weiterentwicklungen.
