\chapter{Fazit}\label{fazit}
Im Rahmen dieser Studienarbeit II wird die in Studienarbeit I entwickelte Desktop-Anwendung erfolgreich erweitert um ein interaktives Verbindungsmanagementsystem.
Mithilfe der implementierten Erweiterungen wird es Nutzern der Anwendung ermöglicht, manuell einen Verbindungspartner aus einer visuell veranschaulichten Liste von anderen verfügbaren Peers im lokalen Netzwerk auswählen zu können.
Von der Entwicklung werden sowohl strukturelle Anpassungen im Backend als auch die Erweiterung des Frontends um ein dynamisches Peer-Auswahlsystem abgedeckt.\\
\\
Die funktionalen und nicht-funktionalen Anforderungen aus Kapitel \ref{anforderungsanalyse} werden durch die Implementierung vollständig umgesetzt.
Im Backend wird eine kontinuierliche Peer-Discovery im nicht verbundenen Zustand (FA-01 & FA-02) implementiert und im Frontend eine Übersicht zur Visualiserung verbindungsbereiter Peers in Form eines Grids mit Buttons (FA-04, FA-04, NFA-08).
Durch die zusätzlichen Implementierungen im Backend, wie die Umsetzung des Verbindungs-Handshake-Protokolls (FA-05), den bidirektionalen Informationsaustausch (FA-06) und die Verbindungsexklusivität (FA-07), wird eine Verbindungsverwaltung innerhalb der Anwendung erschaffen.
Zur manuellen Verbindungstrennung wird über die Umsetzung manueller Disconnects durch einen Button im Frontend (FA-08), Graceful Disconnects (FA-09) und eine Zustandssynchronisation bei Disconnects (FA-10) ermöglicht.
Darüber hinaus werden sowohl im Frontend als auch im Backend der Anwendung nicht funktionale Anforderungen umgesetzt hinsichtlich der Performance (NFA-01 bis NFA-03), Benutzerfreundlichkeit (NFA-04 & NFA-05), Zuverlässigkeit (NFA-06) und Skalierbarkeit (NFA-08).\\
\\
Während der Implementierung wird gezeigt, dass die Map-basierte Datenstruktur zur Eindeutigkeit von Peer-Buttons in dynamischen User-Interface-Updates realisiert werden kann.
Zusätzlich wird mit der Realisierung der Handshake-Protokolle auf Anwendungsebene verdeutlicht, dass neben dem TCP-Handshake zusätzliche Verifikationsschritte notwendig sind für die Kommunikation zweier Peers, damit keine Anwendungen als Peers erkannt werden, die eigentlich nicht erkannt werden sollen.\\
\\
Für zukünftige Erweiterungen des Projekts bestehen verschiedene Möglichkeiten.
Eine kontinuierliche Peer-Discovery auch während bestehender Verbindungen würde es ermöglichen, ohne Trennung der aktuellen Verbindung bereits nach potenziellen neuen Verbindungspartnern zu suchen.
Die Implementierung eines Verbindungsverlaufs könnte dem Nutzer Einblick in vergangene Verbindungen geben.
Zusätzlich könnte ein Mechanismus zur Priorisierung bevorzugter Peers implementiert werden, etwa durch das Speichern häufig genutzter Verbindungen.
Eine weitere sinnvolle Erweiterung wäre die Unterstützung mehrerer gleichzeitiger Verbindungen, wobei die aktuelle Exklusivität (eine Verbindung pro Instanz) aufgehoben werden müsste.\\
\\
Durch die Fertigstellung dieser Arbeit wurde die ursprüngliche Anwendung um wesentliche Funktionalitäten erweitert, die die Nutzerfreundlichkeit und Kontrolle über Netzwerkverbindungen deutlich verbessern.
Die klare Architektur mit Trennung zwischen automatischer Erkennung und manueller Steuerung schafft die Grundlage für weitere Entwicklungen im Bereich dezentraler Peer-to-Peer-Anwendungen.