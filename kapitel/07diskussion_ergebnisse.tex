\chapter{Diskussion von Ergebnissen}\label{diskussion}
Zur Einordnung der entwickelten Algorithmen für die Implementierung der Desktopanwendung werden nun die gewählten Lösungsansätze reflektiert und verglichen mit anderen Ansätzen. 
Zusätzlich werden die Limits und Skalierbarkeit der Software analysiert und alle noch bekannten Probleme dokumentiert.

% Reflexion aktueller Lösung
Mithilfe der in Kapitel \ref{algorithmen} erläuterten Algorithmen zur Implementierung des Frontend und Backends wird eine Lösung entwickelt mit der das Ziel einer Anwendung zur praktischen Umsetzung und Visualisierung des Leser-Schreiber-Problems erreicht werden kann.
Gelöst wird die Kommunikation zwischen Frontend und Backend über einen Message-Channel, was alternativ durch eine REST-Schnittstelle ersetzt werden könnte.
Dies würde zusätzlich die Möglichkeit eröffnen beim Frontend ein Web-Framework einzusetzen im Vergleich zu einem nativ in Go implementierten Frontend.
Zwar wird von den Autoren dieser schriftlichen Arbeit die Systemarchitektur so betrachet, dass für Entwickler mit Hintergrundwissen in der Webentwicklung die Implementierung des Frontends in einem Web-Framework deutlich weniger Arbeit bedeuten würde, trotz des erhöhten Aufwands durch die REST-Schnittstelle.
Jedoch wurde eine Frontend-Lösung entwickelt, die mindestens genauso performant ist, wenn nicht sogar performanter abschneidet.

% Aktuelle Limits und Skalierbarkeit?
Aktuell ist es nur möglich sich mit einer weiteren Backend-Instanz der Anwendung im lokalen Netzwerk zu verbinden, dennoch würde es möglich sein, durch die \gls{glos:nebenläufigkeit} der Prozesse für die Verarbeitung von TCP-Verbindungen im Backend mehrere TCP-Verbindungen zu verschiedenen Hostsystemen aufbauen zu können.
Dies würde jedoch mehr Prüfungen bedingen, damit durch zwei oder mehr Verbindungen keine Konflikte in der Datenverarbeitung verursacht werden.
In den aktuell drei Intervallen, in denen die Nachrichten in den TCP-Paketen versendet und empfangen werden, werden keine merkbar hohen Latenzen oder Auslastungen im Netzwerk beobachtet, weswegen in diesem Bereich der Anwendung kein Limit erkannt werden kann.
Problematisch würde es möglicherweise werden, sobald mehrere Anwendungsinstanz-Paare in hoher Anzahl in einem gemeinsamen lokalen Netzwerk genutzt werden würde bis die jeweilige Bandbreite des Netzwerkes vollständig ausgelastet wäre bzw. überschritten werden würde.

% Aktuelle Probleme und bekannte Bugs?
Zurzeit werden bei der Nutzung der Anwendung keine Bugs erkannt durch bisherige Tests, jedoch wurde bisher nicht absolut jedes Szenario für die Nutzung der entwickelten Anwendung ausprobiert.
Getestet wird während der Entwicklung mit zwei Instanzen der kompilierten Software auf einem Hostsystem und einer TCP-Verbindung über das Loopback-Interface.
Ebenfalls wird separat erprobt, ob zwei verschiedene Geräte in einem lokalen Netzwerk funktionieren, was erfolgreich bestätigt werden konnte.
Bei der Entwicklung wurde zusätzlich auf dem Betriebssystem MacOS getestet, was bei der Kompilierung zu unbekannten und teilweise nicht gelösten Problemen führte.
Demnach wird die Anwendung vorerst nur für Windows und Linux bereit gestellt.