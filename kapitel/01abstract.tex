\chapter*{Abstract}\label{abstract}
\addcontentsline{toc}{chapter}{Abstract}
In einer Welt mit stetig wachsender Vernetzung spielt die effiziente und sichere Interaktion zwischen Computern innerhalb eines Netzwerks eine zentrale Rolle. Besonders in verteilten Systemen stellen gleichzeitige Lese- und Schreibzugriffe auf gemeinsame Datenstrukturen eine Herausforderung dar, da sie leicht zu Inkonsistenzen oder Datenverlust führen können. 

Um dieses sogenannte Leser-Schreiber-Problem besser zu verstehen, wird im Rahmen dieser Arbeit ein Desktop-Programm entwickelt, das die Übertragung von Paketen über das TCP-Protokoll ermöglicht zwischen zwei Instanzen im lokalen Netzwerk.
Die Implementierung erfolgt in der Programmiersprache Go mithilfe des Frameworks Fyne für das Frontend.
Dabei werden durch das Backend der Anwendung Pakete automatische an die andere Instanz versendet, die Anzahl der erhaltenen Pakete gezählt, lokal gespeichert und mithilfe eines Message-Channels ans Frontend gesendet. 
Im Frontend werden die aktuellen Zähler der Pakete angezeigt und durch den gleichzeitigen Zugriff von Frontend und Backend auf den Message-Channel tritt das Leser-Schreiber-Problem auf bzw. wird dessen Lösung visualisiert.
Somit kann es für visuell-lernende Menschen erleichtert werden dieses Problem nachzuvollziehen.

% Bei mehreren Autoren
%%%%%%%%%%%%%%%%%%%%%%%%
\section*{Schreibverteilung}
\begin{table}[H]
\small	
\begin{tabular}{| l | l |}
\hline
Autor   & Kapitel \\ [0.5ex]
\hline
\hline
Theo Leuthardt & Abstract\\
        & Einleitung\\
        & Problemstellung\\
        & Theoretische Grundlagen\\
        & Entwurf\\
        & Algorithmen\\
\hline
Alexander Betke & Diskussion\\
        & Fazit\\
\hline
\end{tabular}
\end{table}
\clearpage
%%%%%%%%%%%%%%%%%%%%%%%%