\chapter*{Abstract}\label{abstract}
\addcontentsline{toc}{chapter}{Abstract}

In verteilten Peer-to-Peer-Systemen stellt die manuelle Steuerung von Netzwerkverbindungen eine zentrale Herausforderung
dar, insbesondere wenn Nutzer gezielt entscheiden sollen, mit welchen Instanzen im lokalen Netzwerk eine Verbindung
hergestellt werden soll.
Während automatische Verbindungsmechanismen die Komplexität reduzieren, wird dadurch die Kontrolle über die
Netzwerktopologie eingeschränkt.
\\
Aufbauend auf der vorangegangenen Studienarbeit I, in der eine Desktop-Anwendung zur Visualisierung des
Leser-Schreiber-Problems entwickelt wird,
wird im Rahmen dieser Arbeit die bestehende Go-Anwendung um ein interaktives Verbindungsmanagementsystem erweitert.
Während in der ursprünglichen Version Verbindungen automatisch mit dem ersten gefundenen Peer aufgebaut werden,
wird nun eine benutzergesteuerte Verbindungsverwaltung implementiert.
\\
Zur Realisierung werden neue Handshake-Protokolle entwickelt, die eine gezielte Verbindungssteuerung ermöglichen.
Im Backend wird ein Mechanismus zur automatischen Peer-Discovery im lokalen Netzwerk implementiert,
der alle verfügbaren Instanzen erkennt und deren Status überwacht.
Im Frontend wird eine Benutzeroberfläche entwickelt, die eine manuelle Auswahl der Verbindungspartner ermöglicht und
den aktuellen Verbindungsstatus visualisiert.
Zusätzlich wird ein Graceful-Disconnect-Mechanismus implementiert, der saubere Verbindungstrennungen gewährleistet.
\\
Durch die klare Trennung zwischen automatischer Netzwerkerkennung im Backend und manueller Verbindungsverwaltung im
Frontend wird volle Kontrolle über Peer-to-Peer-Verbindungen in dezentralen Go-Anwendungen ermöglicht.
Die Erweiterung wird anhand funktionaler und nichtfunktionaler Anforderungen analysiert und
deren Implementierung dokumentiert.


% Bei mehreren Autoren
%%%%%%%%%%%%%%%%%%%%%%%%
\section*{Schreibverteilung}
\begin{table}[H]
\small
\begin{tabular}{| l | l |}
\hline
Autor   & Kapitel \\ [0.5ex]
\hline
\hline
Theo Leuthardt & Abstract\\
        & Einleitung\\
        & Problemstellung\\
        & Anforderungsanalyse\\
        & Theoretische Grundlagen\\
        & Entwurf\\
        & Algorithmen\\
\hline
Alexander Betke & Diskussion\\
        & Fazit\\
\hline
\end{tabular}
\end{table}
\clearpage
%%%%%%%%%%%%%%%%%%%%%%%%