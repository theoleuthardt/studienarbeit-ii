\chapter*{Abstract}\label{abstract}
\addcontentsline{toc}{chapter}{Abstract}

In heutigen verteilten Systemen stellt die benutzerorientierte Herstellung einer Netzwerkverbindung eine 
zunehmende Komplexität dar.
Automatisierungen hinsichtlich des Verbindungsaufbaus reduzieren diese Komplexität erheblich, jedoch wird damit 
auch ein Teil der Kontrolle über die Netzwerktopologie abgegeben.\\
\\
Die in Studienarbeit I entwickelte Desktop-Anwendung mit der Programmiersprache Go zur Visualiserung des Leser-/Schreiber-Problems wird
im Rahmen dieser Arbeit um ein Interaktives Verbindungsmanagementsystem erweitert.
In der vorherigen Version der Desktop-Applikation werden andere Instanzen der Anwendung im lokalen Netzwerk gesucht und es wird eine
Verbindung mit der erst möglichen Instanz automatisch hergestellt.
Mit der Erweiterung im Zuge dieses Projekts wird dem Benutzer ermöglicht selbst eine beliebige Instanz im lokalen Netzwerk auszuwählen
und eine Verbindung aufzubauen.\\
\\
Zur Umsetzung dieser Erweiterung werden eigen entwickelte TCP-Handshake-Protokolle entwickelt auf Basis des TCP-Handshake-Protokolls.
Das Backend wird mit jenen Protokollen zur Discovery anderer Peers ergänzt inklusive neuen Implementierungen für den Verbindungsaufbau.
Zur Anzeige aller aktuell verbindungsbereiten Peers im lokalen Netzwerk wird dem Frontend eine Übersicht hinzugefügt, mit der der Benutzer
einen Peer zum Verbindungsaufbau auswählen kann.
Für die saubere Trennung einer bestehenden Verbindung werden im Backend Graceful Disconnections implementiert.\\
\\

% Bei mehreren Autoren
%%%%%%%%%%%%%%%%%%%%%%%%
\section*{Schreibverteilung}
\begin{table}[H]
\small
\begin{tabular}{| l | l |}
\hline
Autor   & Kapitel \\ [0.5ex]
\hline
\hline
Theo Leuthardt
        & Abstract\\
        & Einleitung\\
        & Problemstellung\\
        & Anforderungsanalyse\\
        & Entwurf - Frontend\\
        & Algorithmen - Frontend\\
        & Fazit\\
        
\hline
Alexander Betke
        & Theoretische Grundlagen\\
        & Entwurf - Frontend\\
        & Algorithmen - Backend\\
        & Diskussion\\
\hline
\end{tabular}
\end{table}
\clearpage
%%%%%%%%%%%%%%%%%%%%%%%%