\chapter*{Abstract}\label{abstract}
\addcontentsline{toc}{chapter}{Abstract}

In verteilten Peer-to-Peer-Systemen stellt die manuelle Steuerung von Netzwerkverbindungen eine zentrale Herausforderung
dar.
Während automatische Verbindungsmechanismen die Komplexität reduzieren, wird dadurch die Kontrolle über die
Netzwerktopologie eingeschränkt.
\\
Aufbauend auf der vorangegangenen Studienarbeit I, in der eine Desktop-Anwendung zur Visualisierung des
Leser-Schreiber-Problems entwickelt wird,
wird im Rahmen dieser Arbeit die bestehende Go-Anwendung um ein interaktives Verbindungsmanagementsystem erweitert.
Während in der ursprünglichen Version Verbindungen automatisch mit dem ersten gefundenen Peer aufgebaut werden,
wird nun eine benutzergesteuerte Verbindungsverwaltung implementiert.
\\
Für die Realisierung werden Handshake-Protokolle erschaffen auf Basis des TCP-Handshake-Protokolls.
Das Backend wird erweitert zur automatischen Peer-Discovery im lokalen Netzwerk,
um alle verfügbaren Instanzen zu erkennen und deren Status zu überwachen.
Die Benutzeroberfläche im Frontend wird erweitert mit einer Visualisierung des aktuellen Verbindungsstatuses und 
eines wweiteren Fensters zur manuellen Auswahl der Verbindungspartner.
Zusätzlich wird ein Graceful-Disconnect-Mechanismus implementiert, der saubere Verbindungstrennungen gewährleistet.
\\
Durch die klare Trennung zwischen automatischer Netzwerkerkennung im Backend und manueller Verbindungsverwaltung im
Frontend wird volle Kontrolle über Peer-to-Peer-Verbindungen in dezentralen Go-Anwendungen ermöglicht.


% Bei mehreren Autoren
%%%%%%%%%%%%%%%%%%%%%%%%
\section*{Schreibverteilung}
\begin{table}[H]
\small
\begin{tabular}{| l | l |}
\hline
Autor   & Kapitel \\ [0.5ex]
\hline
\hline
Theo Leuthardt & Abstract\\
        & Einleitung\\
        & Problemstellung\\
        & Anforderungsanalyse\\
        & Entwurf - Frontend\\
        & Algorithmen - Frontend\\
        & Fazit
\hline
Alexander Betke & Grundlagen\\
        & Entwurf - Backend\\
        & Algorithmen - Backend\\
        & Diskussion\\
\hline
\end{tabular}
\end{table}
\clearpage
%%%%%%%%%%%%%%%%%%%%%%%%