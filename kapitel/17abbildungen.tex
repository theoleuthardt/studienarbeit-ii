\chapter{Abbildungen}\label{ch:abbildungen}

\section{Bilder}\label{ch:bilder}

Hier seht ihr eine supercoole Abbildung ...\\

\begin{figure}[H]
    \centering
    \includegraphics[scale=0.06]{bilder/HWR.png}
    % für die Größe könnt ihr das bild skalieren (scale)
    % die Höhe/Breite fest festlegen (z. B. width = 20em/400pt/etc)
    % oder abhängig von der Seiten-/Textbreite und -höhe skalieren (z. B. width=1.0\linewidth)
    \caption{Ein einzelnes Bild}
    \label{fig:single}
\end{figure}


... gefolgt von einer fast noch cooleren Abbildung, weil die zweite einen Rahmen hat. Die Caption ist über oder unter dem Bild, abhängig davon, ob ihr die \verb|\caption{}| Zeile über oder unter die \verb|\includegraphics[]{}| Zeile schreibt, auch bei Codeausschnitten und Tabellen.\\
In Informatik-Publikationen wird die Caption einer Tabelle „vor“ der Tabelle platziert. Bei Abbildungen, Grafiken, Codeausschnitten, etc. wird es „danach“ eingefügt.\\ % das ist ein direktes Zitat von Frau Monett Diaz

\begin{figure}[H]
    \centering
    \caption[Einzelne Bilder mit Rahmen]{Ein einzelnes Bild mit Rahmen}
    \fbox{\includegraphics[width=0.3\linewidth]{bilder/HWR.png}}
    \label{fig:singlefbox}
\end{figure}


Default ist die Caption auch das, was im Abbildungsverzeichnis angezeigt wird, man kann das aber anpassen (siehe Code).\\
\verb|\caption[Beschreibung im Abbildungsverzeichnis]{Beschreibung im Text}|

Bei ähnlichen Abbildungen oder um Platz zu sparen, können \verb|subfloat|s benutzt werden:\\

\begin{figure}[H]
    \centering
    \caption{Zwei Mal das HWR Logo}
    \subfloat[HWR Logo]{\fbox{\includegraphics[height=0.04\textheight]{bilder/HWR.png}}
    \label{fig:multi1}}
    \subfloat[auch das HWR Logo]{\fbox{\includegraphics[height=0.04\textheight]{bilder/HWR.png}}
    \label{fig:multi2}}
    \label{fig:multiple}
\end{figure}

Mit \verb|wrapfigure| können Bilder auch in den Text eingebunden werden, dazu kein Beispiel, denn ich versteh's nicht.
\clearpage

\section{Code}\label{ch:code}
Das Aussehen der Codeausschnitte kann im Dokumentenkopf in \verb|main.tex| definiert werden.\\
Bei der Nummerierung wird das Kapitel vorgestellt. Warum? Weiß ich nicht. 

Die Beispiele sind aus einem alten PTB und dementsprechend kontextual unwichtig. Von Codeausschnitten mit schwarzem Hintergrund ist abzuraten, da ihr vermutlich Punktabzug für die Lesbarkeit kriegt.\\

\lstset{style=cli}
\begin{lstlisting}[caption=CLI Codebeispiel,label={lst:cli}]
> dotnet ef migrations add "Titel"
\end{lstlisting}
Durch die Anpassung der Listsets kann das Highlighting verbessert werden, siehe \verb|main.tex| etwa Zeile 100.\\

\lstset{style=ps}
\begin{lstlisting}[caption=PowerShell Codebeispiel,label={lst:ps}] 
function CreateSubnetAndDevices {
    param (
        [int]$startId,
        [int]$endId,
        [string]$baseIp,
        [int]$devicesPerSubnet,
        [int]$baseSubnetId = 0
    )
    
    $userId = "USERID"
    $managerId = "MANID"
\end{lstlisting}

