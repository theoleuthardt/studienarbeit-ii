\chapter{Tabellen}\label{ch:tabellen}
Mit Tabellen kann man super viele coole Sachen machen!
\begin{table}[H]
    \centering
    \begin{tabular}{c}
         Zum Beispiel Textpassagen hervorheben\\
         ohne nervige Anpassungen in der Ausrichtung\\
    \end{tabular}
    \label{table:tab1}
\end{table}

\begin{table}[H]
    \centering
    \begin{tabular}{|c|}
         \hline
         Auf ganz viele verschiedene Arten\\
         (ist das cool!)\\
         \hline
    \end{tabular}
    \label{table:tab2}
\end{table}

Hier ist ein Beispiel für eine ganz einfach Tabelle:
\begin{table}[H]
    \centering
    \begin{tabular}{|l|l|l|l|}
         \hline
         & & &  \\
         \hline
         & & &  \\
         \hline
         & & &  \\
         \hline
    \end{tabular}
    \caption{Einfache Tabelle}
    \label{table:tab3}
\end{table}

Ein Beispiel für eine coole Tabelle mit Tabellenkopf:
\begin{table}[H]
    \centering
    \begin{tabular}{|l|l|l|l|}
         \hline
         & & &  \\
         \hline
         \hline
         & & &  \\
         \hline
         & & &  \\
         \hline
    \end{tabular}
    \caption{Tabelle mit Tabellenkopf}
    \label{table:tab4}
\end{table}

Eine noch coolere Tabelle mit Legende:
\begin{table}[H]
    \centering
    \begin{tabular}{|l|}
         \hline
         \textbf{Ausrichtung}\\
         \hline
         Beispieltext\\
         \hline
    \end{tabular}
    \begin{tabular}{|l|c|r|}
         \hline
         links & mitte & rechts\\
         \hline
         damit man & die Ausrichtung & richtig sieht.\\
         \hline
    \end{tabular}
    \caption{Tabelle mit Legende}
    \label{table:tab5}
\end{table}

Man kann mehrzeilige Zellen erzeugen, in dem man (1) kein \verb|\hline| zwischen die Zeilen schreibt, oder (2) durch die Verwendung von \verb|multirow|.
\begin{table}[H]
    \centering
    \footnotesize
    \begin{tabular}{|l|l|l|l|}
         \hline
         (1) & & &  \\
         \hline
         \hline
         Um den      & zu sehen,  & Sinn, in & zu gucken, \\
         Unterschied & ergibt es  & den Code & LG. \\
         \hline
    \end{tabular}
    \begin{tabular}{|l|l|l|l|}
        \hline
         (2) & & & \\
        \hline
        \hline
        \multirow{2}{5em}{Um den Unterschied} & \multirow{2}{4em}{zu sehen, ergibt es} & \multirow{2}{4em}{Sinn, in den Code} & \multirow{2}{4.5em}{zu gucken, LG.} \\
        \\ 
        \hline 
    \end{tabular}
    \caption{Tabelle mit verbundenen Zeilen} (Ich hab keine Ahnung, warum die vertikalen Striche fehlen.)
    \label{table:tab6}
\end{table}
\clearpage

Mit \verb|multicolumn| kann man Zellen horizontal verbinden:
\begin{table}[H]
    \centering
    \begin{tabular}{|l|l|l|}
        \hline
        \multicolumn{3}{|l|}{Das kann dann zum Beispiel}\\
        \hline
        so aussehen,       & \multicolumn{2}{|l|}{wobei ich mir ehrlich}\\
        \hline
        \multicolumn{3}{|l|}{unsicher bin, ob man multicolumn mit multirow kombinieren kann.}\\
        \multicolumn{3}{|l|}{Deshalb sind diese Zeilenumbrüche „manuell“.}\\
        \hline
        Mehr Beispieltext   & fällt mir an dieser Stelle & auch gar nicht mehr ein.\\
        \hline
        Ich denke in diesem & Maß gibt es auch           & eigentlich keine\\
        \hline
        \multicolumn{3}{|c|}{Anwendungsfälle, zumindest nicht beim}\\
        \hline
        wissenschaftlichen  & Das ist aber eine          & falls ihr mal einen \\
        Schreiben.          & super Vorlage,             & Projektsteckbrief machen müsst.\\
        \hline
        \multicolumn{2}{|l|}{Viele liebe Grüße!}         & :D\\
        \hline
    \end{tabular}
    \caption{Tabelle mit verbundenen Spalten}
    \label{table:tab7}
\end{table}

Und wenn man das alles kombiniert, kann man zum Beispiel supercoole Matrizen in seine Arbeit einbinden:
\begin{table}[H]
    \begin{tabular}{|c|}
        \hline
        \multirow{2}{*}{Repository} \\
        \\
        \hline
        \hline
        \multirow{2}{*}{Card} \\
        \\ 
        \hline \hline
        \multirow{2}{*}{Verification} \\
        \\ 
        \hline
    \end{tabular}
    \begin{tabular}{|l|}
        \hline
        \multirow{2}{*}{Profile} \\
        \\
        \hline
        \hline
        devlocal\\ \hline
        dev/test/prod\\ \hline
        \hline
        devlocal\\ \hline
        dev/test/prod\\ \hline
    \end{tabular}
    \begin{tabular}{|c|c|c|c|c|c|c|}
        \hline
        Spring&Server&Data&Test&Private&Public&Website\\
        Boot&Port&Source&User&Key Path&Key Path& URL\\
        \hline
        \hline
        X & X & X & X & X & X & X \\ \hline
        X &   & X &   & X & X & X \\ \hline % ihr könnt einrücken für Übersichtlichkeit
        \hline
        X&X&&X&&&\\ \hline
        X&&&&X&&\\ \hline                   % aber wenn ihrs nicht macht, kommt das gleiche dabei raus
    \end{tabular}
    \caption[Supercoole Matrix!!!]{Notwendige Properties für die jeweiligen Profile}
\end{table}