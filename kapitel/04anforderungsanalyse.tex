\chapter{Anforderungsanalyse}\label{anforderungsanalyse}
Mit der Erweiterung der vorhanden Implementierung aus Studienarbeit I werden Definitionen von Anforderungen gefordert.
Damit wird eine Grundlage gebildet, auf der der Entwurf und die Implementierung des Verbindungsmanagementsystems umgesetzt werden kann.
Es wird als notwendig betrachtet, dass alle Grundfunktionalitäten aus der Studienarbeit I beibehalten werden.
Dazu wird auf funktionaler Seite das automatische Erkennen von Peers im lokalen Netzwerk gezählt, sowie die Nutzung des Portbereichs 50500 bis 50600, das gegenseitige Versenden von TCP-Paketen, die Anzeige und das Monitoring des aktuellen Verbindungsstatuses, die Visualiserung des Leser-/Schreiber-Problems und die Channel-basierte Kommunikation zwischen Frontend und Backend.
Bei den nicht funktionales Anforderungen werden die Plattformunabhängigkeit der Anwendung, UI-Skalierung und der Dark-Mode als vorhanden gesehen.\\
In den folgenden Teilabschnitten werden die Anforderungen in funktionale und nicht funktionale unterteilt und darin jeweils in kategorisierte Untergruppen.
Jede definierte Anforderung wird mit einer eindeutigen Indentifikationsnummer notiert für die spätere Referenzierung in Kapitel \ref{entwurf} und \ref{algorithmen}.

\section{Funktionale Anforderungen}\label{funktionale_anforderungen}
In folgender Auflistung werden die neuen funktionalen Anforderungen beschrieben für das erwartete Verhalten der Desktop-Anwendung.

\subsection{Peer-Discovery}
\begin{description}
    \item[FA-01: Peer-Liste-Verwaltung] \hfill \label{FA-01} \\
    Zusätzlich zur automatischen Erkennung von Peers im lokalen Netzwerk soll das Backend eine Liste an erkannten Peers führen und diese regelmäßig bei jeder neuen Suche aktualisieren.

    \item[FA-02: Discovery-Handshake] \hfill \label{FA-02} \\
    Damit auch nur valide Peers erkannt werden, soll vor dem eigentlichen Verbindungs-Handshake eine Erkennungs-Handshake implementiert werden in Form von: \texttt{DISCOVER\_SYN} / \texttt{DISCOVER\_ACK}.
    Dies soll zur Prüfung dienen, ob der gefundene Peer verfügbar für eine anschließende Verbindung ist.
\end{description}

\subsection{Verbindungsverwaltung}
\begin{description}
    \item[FA-03: Manuelle Peer-Auswahl] \hfill \label{FA-03} \\
    Bei der Nutzung der Anwendung soll der Benutzer über eine grafische Benutzeroberfläche eine Übersicht über alle aktuell erkannten Peers erhalten, mit denen potentiell eine Verbindung aufgebaut werden kann.

    \item[FA-04: Peer-Auswahl-Ausblendung im verbundenen Zustand] \hfill \label{FA-04} \\
    Wird eine Verbindung hergestellt mit einem anderen Peer, so soll die Peer-Auswahl-Übersicht aus Anforderung \ref{FA-03} ausgeblendet werden.
    
    \item[FA-05: Verbindungs-Handshake-Protokoll] \hfill \label{FA-05} \\
    Für das Verbinden mit einem Peer soll ein Handshake-Protokoll implementiert werden mit \texttt{CONNECT\_REQ:<ip>:<port>} und \texttt{CONNECT\_OK} Nachrichten.

    \item[FA-06: Bidirektionale Informationsaustausch über die Verbindung] \hfill \label{FA-06} \\
    Bei einer Verbindungsanfrage über die \texttt{CONNECT\_REQ} Nachricht soll die IP-Addresse und der Port des Verbindungsinitiators mit versendet werden, damit auch der Empfänger die Information erhält, wer mit ihm eine Verbindung aufbauen möchte.

    \item[FA-07: Verbindungsexklusivität] \hfill \label{FA-07} \\
    Wird zwischen zwei Peers eine Verbindung erfolgereich aufgebaut und erhalten, sollen Verbindungsanfrage-Nachrichten anderer Peers währenddessen ignoriert werden.
\end{description}

\subsection{Verbindungstrennung}
\begin{description}
    \item[FA-08: Manueller Disconnect] \hfill \label{FA-08} \\
    Dem Benutzer der Anwendung soll es über eine Schaltfläche ermöglicht werden, eine bestehende Verbindung zu einem anderen Peer trennen zu können, ohne die Anwendung schließen zu müssen.

    \item[FA-09: Graceful-Disconnects] \hfill \label{FA-09} \\
    Wird eine bestehende Verbindung zu einem anderen Peer manuell unterbrochen durch den Benutzer, soll eine \texttt{DISCONNECT} Nachricht erst an den Peer gesendet werden für die Verbindungsauflösung.

    \item[FA-10: Zustandssynchronisation bei Disconnect] \hfill \label{FA-10} \\
    Nachdem durch den Benutzer die Verbindung zu einem anderen Peer getrennt wird, sollen beide Instanzen ihre Zustände wie der Verbindungsstatus, die Adresse des verbundenen Peers und die Queue-Werte zurücksetzen auf ihre Initialwerte.
\end{description}

\section{Nichtfunktionale Anforderungen}\label{nichtfunktionale_anforderungen}
Im Folgenden werden die nicht funktionalen Anforderungen definiert zur Hervorhebung qualitativer Eigenschaften der Anwendung.

\subsection{Performance}
\begin{description}
    \item[NFA-01: Discovery-Geschwindigkeit] \hfill \label{NFA-01} \\
    Ein vollständiger Discovery-Scan (localhost + Subnetz) soll innerhalb von maximal 5-7 Sekunden abgeschlossen sein.

    \item[NFA-02: UI-Aktualisierungsrate] \hfill \label{NFA-02} \\
    Die Benutzeroberfläche muss im Intervall von 100ms mit aktuellen Zustandsinformationen aus dem Backend aktualisiert werden.

    \item[NFA-03: Handshake-Timeout] \hfill \label{NFA-03} \\
    Discovery-Handshakes sollen einen Timeout von 2 Sekunden haben und die Verbindungs-Handshakes einen Timeout von 1 Sekunde.
\end{description}

\subsection{Benutzerfreundlichkeit}
\begin{description}
    \item[NFA-04: Intuitive Bedienung] \hfill \label{NFA-04} \\
    Die Auswahl von Verbindungspartnern und das Trennen der aktuellen Verbindung soll über jeweils einen Klick für den Benutzer erledigt werden können.

    \item[NFA-05: Identifikation des Verbindungspartners vor dem Verbindungsaufbau] \hfill \label{NFA-05} \\
    Jeder der angezeigten Schaltflächen für einen möglichen Verbindungspartner soll die IP-Adresse und den Port des Peers enthalten.
\end{description}

\subsection{Zuverlässigkeit}
\begin{description}
    \item[NFA-06: Fehlertoleranz] \hfill \label{NFA-06} \\
    Die Anwendung soll bei Netzwerkfehlern oder beim Abbruch der Verbindung automatisch in einen weiterhin konsistenten Zustand bleiben.
\end{description}

\subsection{Skalierbarkeit}
\begin{description}
    \item[NFA-08: Dynamisches Anzeige der verbindungsbereiten Peers] \hfill \label{NFA-08} \\
    Die Anzahl der Schaltflächen zur manuellen Verbindung mit anderen Peers soll nicht durch die Fenstergröße begrenzt sein.
    Das bedeutet, neue Schaltflächen sollen zur Übersicht hinzugefügt werden, jedoch soll das Fenster nicht linear mit der Anzahl der anzuzeigenden Buttons größer werden.
\end{description}