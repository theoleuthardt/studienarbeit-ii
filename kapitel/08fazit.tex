\chapter{Fazit}\label{fazit}
Im Rahmen dieser Arbeit wurde erfolgreich eine Desktop-Applikation entwickelt, mit der das Leser-Schreiber-Problem visuell erlebt werden kann.
% Erreichte Ziele vs ursprüngliche Ziele
Mithilfe der Software wurde erreicht eine TCP-Verbindung zwischen zwei Instanzen aufbauen zu können und automatisch TCP-Pakete in drei verschiedenen Arten auszutauschen, die die drei Bufferlänger darstellen.
Die gesamte Anwendung wurde in Go implementiert und durch die Entwicklung des Frontends und Backends in der gleichen Programmiersprache wurde es ermöglicht höhere Komplexität im Projekt dadurch vermeiden zu können.
Die genannten Anforderungen aus dem Kapitel \ref{problemstellung} wurden praktisch umgesetzt mithilfe der Struktur aus Kapitel \ref{entwurf} und der Grundfunktionalitäten der Software aus Kapitel \ref{algorithmen}. \\
\\
% Abschluss mit kleiner Reflexion
Während der Bearbeitung des Projekts wurden in der schriftlichen und praktischen Arbeit die Konzepte der \gls{glos:nebenläufigkeit}, des ISO-OSI-Modells aus Kapitel \ref{osimodell} und die Netzwerkprogrammierung in Go auf Basis des TCP-Protokolls aus Kapitel \ref{tcp} verinnerlicht und angewendet.
Vor allem während der praktischen Umsetzung wurde bemerkbar wie vergleichsweise einfach es mittlerweile möglich ist, innerhalb von einer Go-Datei ein Backend aufzusetzen, was auf einem Port des ausführenden Hosts hört und Netzwerkpakete versenden und empfangen kann.\\
\\
% Anregungen für die Zukunft / Ideen für Studienprojekt II
Für die Zukunft des Projekt würde es möglich sein in Kapitel \ref{problemstellung} nicht genannte und nur in Erwägung gezogene Anforderungen der Software hinzuzufügen.
Diese könnte in Form eines visuellen Netzwerk-Browsers implementiert werden, mit dem der Nutzer sich manuell und gezielt mit einer anderen bestimmten Instanz im lokalen Netzwerk per TCP verbinden kann und auch manuell selbst bestimmen kann welche Art von TCP-Paket versendet werden soll.
Außerdem könnte es auch möglich sein die Länge des Buffers, der durch das Versenden des Pakets entsteht, selbst zu bestimmen.
Diese und weitere Erweiterungen werden in der Fortführung des Projekts näher betrachtet.
Jedoch wurde durch die Fertigstellung dieser praktischen Arbeit eine Software entwickelt, mit der das Leser-Schreiber-Problem vertieft werden kann, Netzwerkfunktionalitäten in Go erforscht wurden und die Theorie in die Praxis umgesetzt wurde für ein besseres Verständnis des Problems und dessen Lösungen.