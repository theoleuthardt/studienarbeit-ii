\chapter{Problemstellung}\label{problemstellung}
Das mit der Studienarbeit I implementierte Softwaresystem in Form einer Desktop-Anwendung ermöglicht es TCP-Verbindungen zwischen zwei Instanzen innerhalb eines lokalen Netzwerks vollautomatisch herzustellen.
Erkennt das Backend einer Instanz eine andere wird direkt eine Verbindungsanfrage gesendet.
Die Automatisierung dieses Prozesses führt zu den drei folgenden Beschränkungen:

\begin{enumerate}
    \item Es kann nicht bestimmt werden, mit welchem Peer die Verbindung aufgebaut wird.
    \item Es existiert keine Funktion zum manuellen Trennen einer bestehenden Verbindung.
    \item Der Anwender erhält keinerlei Information darüber, welche weiteren Peers im Netzwerk verfügbar sind oder mit welchem konkreten Peer aktuell kommuniziert wird.
\end{enumerate}

\\
Falls mehr als zwei Instanzen in einem lokalen Netzwerk ausgeführt werden, ist es mehr Zufall als gewolltes Szenario der Benutzer welche Instanz
mit welcher anderen eine Verbindung eingeht.
In der Regel sendet eine Instanz bei beispielsweise drei Instanzen in einem lokalen Netzwerk an eine zweite eine Verbindungsanfrage und verbindet sich mit dieser unabhängig davon,
ob dies vom Nutzer so gewollt ist.
Zum Wechseln des Peers mit dem die Verbindung eingegangen werden soll muss die eigene Instanz vollständig beendet werden.
Nur so ist es bisher möglich die Verbindung zu trennen.\\

\section{Problemrelevanz}\label{problemrelevanz}
Die manuelle Verbindungssteuerung wird in modernen verteilten Systemen wie Microservices oder verteilten Netzwerken über mehrere Rechenzentren als essentiell angesehen.
In manchen Szenarien solcher Produktivsysteme wird die bewusste Auswahl des Verbindungspartners erfordert für das Trouble-Shooting fehlerhafter Teilkomponenten oder die Kommunikation mit eindeutigen Services.\\
Im Falle der Visualisierung des Leser-/Schreiber-Problems zum Erlernen für Studierende ist es notwendig verschiedene Szenarien durcharbeiten zu können zum besseren Verständnis des Problems, wodurch das spezifische Auswählen des Kommunikationspartners benötigt wird.
Wird ein Anwendungsbeispiel wie die Verbindung mit Testinstanzen für eine Entwicklungs- bzw. Testumgebung betrachtet, wird die zu aufwendende Zeit für das Testen durch das wiederholte Neustarten der Anwendung erhöht, wodurch ein ineffizienterer Testablauf verursacht werden kann.

\section{Fehlende Funktionalitäten}\label{wasfehlt}
Nach einer Analyse der bestehenden Implementierung der Desktop-Anwendung werden sowohl funktionale als architektonische Defizite in der Frontend- und Backend-Schicht ersichtlich.
Architektonisch wird eine Diskrepanz erkannt zwischen dem Informationsgehalt im Backend über die gefundenen Peers im Backend durch automatische Peer-Discovery und die fehlenden Informationen darüber im Frontend für eine Übersicht des möglichen Verbindungspartner.\\
\\
In der Backend-Schicht der Implementierung wird automatisch nach anderen Peers im lokalen Netzwerk gesucht mit den Ports 50500 bis 50600.
Nach einer erfolgreichen Verbindung mit einem anderen Peer, wird aber der Scan-Algorithmus beendet und erst wenn die Anwendung neu gestartet wird, wird erst wieder nach neuen Peers gesucht.
Sollte also ein Peer nicht auch schon verbindungsbereit im selben Netzwerk vorhanden sein, kann eine Verbindung erst nach einem Anwendungsneustart hergestellt werden.
Außerdem werden durch das Backend zwar bei der Verbindungstrennung die grundlegenden Handshakes unterstützt durch \texttt{CONNECT\_REQ} und \texttt{CONNECT\_OK} Nachrichten, jedoch wird keine Implementierung bereitgestellt für Graceful-Disconnects zur vollständig koordinierten Verbindungstrennung.\\
\\
Zusätzlich wird dem Nutzer im Frontend nicht ersichtlich gemacht, welche anderen Peers gerade zur aktuellen Laufzeit verbindungsbereit vorhanden sind.
Ebenfalls wird keine Benutzerinteraktionsmöglichkeit angezeigt wie eine Schaltfläche zum manuellen Trennen einer aktuell vorhandenen Verbindung.
Durch diese beiden fehlenden Funktionalitäten wird eine Informationslücke für einen kontrollierten Anwendungsablauf dargestellt.